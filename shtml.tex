\documentclass{article}
\usepackage{url}
\title{Statically-Typed HTML Generation}
\author{David LaPalomento}
\date{December 2011}
\begin{document}
\maketitle
When the world wide web was first introduced, it was designed as a system to share simple documents among researchers working around the world. As it has grown from an academic curiosity to a ubiquitious part of our culture, that original model of simple documents has been expanded almost beyond recognition. Modern websites are often generated on-the-fly, stitching together data from geographically distributed data sources and other web services. A web server that simply slings static files across the network is often insufficent to create the complex interactions desired by their creators. To create these more individualized experiences, web authors often turn to template languages: tiny scripts designed to stitch together the dynamic and static portions of an HTML document. Template languages can be awkward to work with, however. The interface between the templates and the general-purpose programming language that run the rest of the web server often provide no automated mechanisms to ensure correct data is being fed to the template. Templating languages are almost exclusively dynamically typed as well, which may not fit with how some programmers or teams prefer to design their applications.

SHTML takes a different approach to templating in order to address some of these concerns. Instead of trying to find a safe way of passing data from a general-purpose programming language to a specialized template language, SHTML compiles regular HTML into a data structure in Scala, a JVM-based programming language popular with web application developers. This data structure can then be manipulated and combined with other data sources just as in traditional templating systems. Unlike traditional template systems, however, SHTML templating allows developers to use the type system in Scala to ensure that all the data necessary to generate a page is well-formed and available at runtime. It gives programmers the same freedom to organize and verify a template's execution as they have elsewhere in the application.

SHTML is designed to work as a plugin for SBT, a Scala-based build tool. SBT has a mechanism to automatically download and install most plugins based on a project's declared dependencies but SHTML is not currently hosted on any public plugin repositories. Instead, SHTML should be built and published to the local dependency cache and then it can be pulled in for new projects. The SHTML plugin itself can be downloaded from its Github repository page at \url{https://github.com/dmlap/shtml}. Once it has been downloaded somewhere convenient, it can be built and published locally through SBT by typing:

\begin{verbatim}
sbt publish-local
\end{verbatim}

SBT should handle fetching Scala and any libraries necessary to use SHTML. To use SHTML in an SBT project, you must declare it in the SBT plugin definition for your project and include its settings. To do that, create or open \begin{tt}project/plugins.sbt\end{tt} and add the following line:

\begin{verbatim}
addSbtPlugin("com.github.dmlap" % "shtml-plugin" % "0.1-SNAPSHOT")
\end{verbatim}

Then, include \begin{tt}seq(shtmlSettings: \_*)\end{tt} in your regular project definition, build.sbt if you're using the SBT configuration DSL. In order to actually use template-generated objects, you'll also need to include the SHTML runtime library as a compile-time dependency.

All of these steps can be reviewed in more depth through the SHTML example project, available at \url{https://github.com/dmlap/shtml-example}. It includes code that configures the SHTML plugin, generates Scala source code from an example HTML template and then constructs a webpage after modifying and expanding upon it. The template itself is in \begin{tt}src/main/webapp/example.html\end{tt}. It is worth noting this template is a completely valid HTML document that can be loaded into a web browser independently of the templating engine. Further, SHTML is designed to work even with less-conformant HTML like that often found on real websites. Self-closing and implicit tags are supported and transformed into well-formed document trees during the template compilation process.

Template manipulation is performed through an update function and element level callbacks. Inside example.html, there is a script element with the \begin{tt}shtml\end{tt} type attribute. The contents of this element are a sequence of CSS-selector statements paired with type name declarations. For instance, the element with the id attribute equal to \begin{tt}description\end{tt} is associated with the name \begin{tt}Description\end{tt}. This name is translated into a Scala type and singleton object during the compilation process. That means they must be unique within the package that the template is being compiled into and can be referred to by that name by code elsewhere in the project. If any type assignments of this sort are present in a template, an update function will be generated. This function takes as arguments other functions which manipulate the specified element of the template and then stitches together their output to build the final page. For a web developer who wants to create a dynamic page, this means they will define the portions of the template to be modified through an SHTML script element and then call a generated update function at some point with manipulation callbacks for all those elements. Examples of updating a template can be found in \begin{tt}src/main/scala/ShtmlExample.scala\end{tt}.

Static types are a valuable and much-appreciated tool for many web application developers. Unfortunately, the benefits of compile-time typechecking can not easily be brought to bear on dynamic web page generation with many of the existing templating systems in use today. SHTML attempts to bring those advantages into web application development through generating code in a general-purpose programming language that can be manipulated just like a typical dynamic HTML template. By integrating it with an existing build tool for Scala, it is very easy to incorporate into existing Scala-based web applications. Dynamic HTML templates are a standard tool for many web applications and are certainly a proven way to build complex websites. For those projects that desire or could benefit from the added security of compiler-assisted template generation, SHTML is a compelling alternative. 
\end{document}